**PROJECT DESCRIPTION**

**PROJECT NAME:** HomePlantyGrower: Automated Greenhouse - Optimizing Plant Cultivation through Modern Technology

**PREPARED BY:** Jakob Hellmuth, Benedikt Trenz, Nico Rast, Massimo Russo

**TITLE:** Microcomputertechnik Projekt

**DATE:** 23.02.2024

**EMAIL:** hellmuth.jakob-it22@it.dhbw-ravensburg.de

**PROJECT OVERVIEW**

Welcome to the project description of our exciting project - the HomePlantyGrower. The innovative HomePlantyGrower aims to be an automated greenhouse, with the goal of revolutionizing private plant cultivation through the use of state-of-the-art technologies.

**PURPOSE / GOALS**

The main goal of this project is to develop and implement an automated indoor greenhouse that allows users to care for and grow crops with minimal manual effort. We aim to increase the efficiency and productivity of plant cultivation while minimizing resource consumption. The target audience for this greenhouse can be very diverse: from students in dormitories to seniors in the city who want to fulfill their dream of having their own garden.

**OBSTACLES**

- Integration of hardware components: Integrating various sensors, the control unit, and actuators such as the pump can lead to technical difficulties and compatibility issues.
  
- Enclosure: Building the enclosure may pose technical challenges due to the limited experience of the developing engineers with the selected materials.

**INDUSTRY / MARKET RISK FACTORS**

Potential competitors:
- The BOSCH SmartGrow Group
- Click & Grow Smart Garden 9 Pro
- Northpoint Plant Lamp
- Prêt à Pousser Nano Garden

While these competitors are mostly more energy-efficient than our product and require less maintenance, they mostly have manual controls and are not as smart as our HomePlantyGrower will be.

**BUDGETARY RISK FACTORS**

- Increase in material procurement costs: Rising material costs can strain the project's budget and lead to unforeseen expenses.
  
- Miscalculations: Unforeseen expenses that were not considered or forgotten in the original budget.
  
- Assembly/testing problems: Components could be damaged during assembly or testing, leading to further expenses.

**HARDWARE COMPATIBILITY**

The project involves integrating various hardware components such as sensors, Arduino, and actuators. The compatibility of these hardware components is crucial for the smooth operation of the automated greenhouse. We will ensure that all hardware components are compatible with each other and work seamlessly together to enable the desired functions. This includes selecting high-quality hardware components, compatibility with the software platform used, as well as comprehensive testing and validation during the development process. By ensuring high hardware compatibility, we aim to maximize the reliability and efficiency of the automated greenhouse and minimize potential disruptions or incompatibilities.

**SOFTWARE EMPLOYED**

- Embedded Software for Arduino: We will use custom embedded software to program the Arduino in the greenhouse and control various systems such as irrigation, lighting, and ventilation.

- Remote Access Software: We will provide users with a way to access the greenhouse remotely via mobile devices. Here, users can read the current sensor values online.

**TIMELINE / MILESTONES**

The planned milestones include completing the conceptual phase, developing the prototype, extensive testing and adjustments, and market launch. We plan to complete the project within 3 months from the start of the project.

See Gantt chart (Management and Development Plan/Schedule).

**DEPLOYMENT / DISTRIBUTION**

After completing the development phase, the greenhouse will be deployed and distributed in several steps:

1. The prototype will be presented and extensively tested.
2. Implementation at selected customers.
3. Based on the results of the pilot project, the greenhouse will be prepared for broader deployment.
4. The greenhouse will be distributed to end customers through various distribution channels.

Through this deployment and distribution process, we aim to efficiently and successfully bring the automated greenhouse to market and provide users with a premier solution for plant cultivation.

**TESTING**

Quality assurance and testing play a crucial role in the development process of the automated greenhouse. We conduct comprehensive tests to ensure that the system functions flawlessly and meets the requirements of our users. The testing phase includes:
  
1. Functionality tests
2. Integration tests
3. Security tests
4. User-friendliness tests

**COST STRUCTURE**

The costs consist of material costs for the prototype, as well as the costs incurred by developing the project. (Calculated based on the number of requirements, costs naturally fall away for our case, as only one prototype is being built.)
